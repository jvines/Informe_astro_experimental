\documentclass[a4paper, 11pt, spanish]{article}

% ---------------- Paquetes de formato.
\usepackage[spanish]{babel} % Para codificar el texto.
\usepackage[top=70mm, bottom=30mm, left=18mm, right=18mm]{geometry} % Para modificar el tamaño de las hojas.
\usepackage{fancyhdr} % Para poner la institución, dpto y curso arriba.
\usepackage{parallel} % Para escribir en columnas (Poner los integrantes a la derecha).
\usepackage[firstpage=false]{background} % Para poner el logo de la U arriba en todas las páginas.
\usepackage{enumerate} % Para poder enumerar como a), b), etc.
\usepackage[utf8]{inputenc} % Para usar acentos en vez de \'.

% ---------------- Paquetes graficos.
\usepackage{graphicx} % remove the demo option.
\usepackage{tikz}
\usepackage{caption}
\usepackage{subcaption} % Para usar sub-figuras

% ---------------- Paquetes matematicos.
\usepackage{amsmath} % Para poder hacer N^o y se vea bonito.
\usepackage{amsthm} % Fonts matematicos.
\usepackage{amssymb} % Para usar \therefore
\usepackage{commath} % Para usar \abs y \norm

% ---------------- Paquetes 'computines'.
\usepackage{listings} % Para escribir codigo y que se vea bonito.
\usepackage[% Descomentar las opciones a usar.
spanish,
%boxed, % Encierra los algoritmos en un cuadro.
boxruled, % Encierra los algoritmos en un cuadro colocando el titulo al comienzo.
%ruled, % Coloca una linea al comienzo y otra al final del algoritmo. El titulo de este queda al comienzo del algoritmo.
%algoruled, % Lo mismo que el anterior pero mas espaciado.
%tworuled, % Como ruled pero sin una linea al comienzo.
%algochapter, % Los algoritmos se enumeran segun capitulo.
%algopart, % Los algoritmos se enumeran por partes.
%figure, % Los algoritmos son considerados figuras (y por ende salen en \listoffigures).
%linesnumbered, % Enumera las lineas.
longend % Los end son para cada ciclo, por ejemplo endif para los if-else.
]{algorithm2e}

% ---------------- Paquetes miscelaneos.
%\usepackage{lipsum} % Para hacer placeholders.
\usepackage{bohr} % Para dibujar atomos.

% ---------------- Comando mas corto para insertar figuras. Ojo que deben estar guardadas en ./img/
% \fig{name}{width}{height}{caption}
\newcommand{\fig}[4]{%
	\begin{figure}[!htbp]
		\centering
		\includegraphics[width=#2, height=#3]{img/#1}
		\caption{#4}
	\end{figure}
}
% ejemplo:
%			\fig{nombre_imagen.png}{10cm}{5cm}{Titulo de la imagen}


%\begin{figure}[!ht]
%\centering
%\begin{subfigure}{.5\textwidth}
%  \centering
%  \includegraphics[width=10cm, height=8cm]{img/curva0f1.pdf}
%  \caption{Curva de nivel 0 de $F_{1}$.}
%  \label{fig:sub1}
%\end{subfigure}%
%\begin{subfigure}{.5\textwidth}
%  \centering
%  \includegraphics[width=10cm, height=8cm]{img/curva0f2.pdf}
%  \caption{Curva de nivel 0 de $F_{2}$.}
%  \label{fig:sub2}
%\end{subfigure}
%\caption{Curvas de nivel 0 de $F_{1}$ y $F_{2}$}
%\label{fig:test}
%\end{figure}


% ---------------- Comando para hacer itemes
% \Solution{pregunta}{solucion}
\newcommand{\Solution}[2]{%
	\item #1 \vspace{0.2cm}
	\textbf{Soluci\'on:} #2
}
% \Demonstration{pregunta}{demostracion}
\newcommand{\Demonstration}[2]{%
	\item #1 \vspace{0.2cm}
	\begin{proof}
		#2	
	\end{proof}
}

% ---------------- Opciones de algortihm2e
\SetKw{KwRequire}{Require:}

% ---------------- Opciones de background.
\SetBgColor{black}
\SetBgScale{1}
\SetBgOpacity{1}
\SetBgAngle{0}
\SetBgContents{%
	\begin{tikzpicture}[remember picture,overlay]
		\node at (-8.0,0.746\textheight) {\includegraphics[height=18mm,width= 0.155\textwidth]{img/LogoUIngenieria.png}};
	\end{tikzpicture}
}

% ---------------- Creacion de institucion, departamento y curso.
\fancyheadoffset[L]{-2cm}
\fancyhead[L]{\footnotesize{\textbf{\textsf{Universidad de Chile \\ Facultad de Cs. F\'isicas y Matem\'aticas \\ Departamento de F\'isica \\ FI3104-1: M\'etodos Num\'ericos para la Ciencia e Ingenier\'ia. }}}}
\renewcommand{\headrulewidth}{0pt}
\setlength{\voffset}{-3cm}

\pagestyle{fancy} % Estilo de las páginas

\begin{document}

\pagenumbering{gobble} % Quita el numero de las paginas (y las resetea a 1)

\clearpage

\thispagestyle{fancy}
\vspace*{6.5cm} % Espacio vertical para posicionar bien el título (En una de esas esto se puede optimizar para que no sea tan a la fuerza bruta).

% ---------------- Titulo.
\begin{center}
	\Large{\textbf{\textsf{Tarea $\text{N}^\text{o}$3}}} \\
	\huge{\textbf{\textsf{Interpolaci\'on de Polinomios.}}}
\end{center}

\vspace*{5.5cm}

% ---------------- Integrantes, profes, etc.
\begin{Parallel}{1cm}{7.5cm}
	\ParallelRText{%
		\begin{flushright} % Tira el texto hacia la derecha.
			\large{%
				\textsf{%
					\begin{tabular}{rl}
%						Integrantes: &
%							\begin{tabular}[t]{@{}l@{}}
%								Integrante 1. \\
%								Integrante 2.
%							\end{tabular} \\
						& Jos\'e Ignacio Vines. \\
						Profesor: & 
							\begin{tabular}[t]{@{}l@{}}
						 		Valentino Gonz\'alez.
							\end{tabular} \\
						Auxiliares: &
							\begin{tabular}[t]{@{}l@{}}
								Mario Aguilar. \\
								Ignacio Armijo. \\
								Mar\'ia Constanza Flores. \\
							\end{tabular} \\	
%						Ayudantes: &
%							\begin{tabular}[t]{@{}l@{}}
%								Ayudante 1. \\
%								Ayudante 2.
%							\end{tabular} \\		 
					\end{tabular}
					Fecha: \today
				}
			}
		\end{flushright}
	}
\end{Parallel}

\clearpage

\pagenumbering{arabic} % Numeros de pagina Arabicos (y los resetea a 1)

\newpage

\tableofcontents % Indice. Descomente para usar.
\listoffigures % Lista de figuras. Descomente para usar.
%\listoftables % Lista de tablas. Descomente para usar.

\newpage

\section{Introducci\'on}
La fotometr\'ia es el \'area de la ciencia donde se mide la luz, es decir, la cantidad de fotones que llegan a partir de una fuente. En particular, en astronom\'ia se cuantifican los fotones que llegan por estrella. Una de las razones por las que esto se estudia es para buscar exoplanetas: al tomar la luz que proviene de una estrella y medirla peri\'odicamente por un tiempo, uno forma una curva de luz: si en esa curva hay decrecimientos notorios (y peri\'odicos, en teor\'ia), puede ser porque hay un planeta orbitando que tapa la luz proveniente de la estrella. 

Hay varias t\'ecnicas de fotometr\'ia; para este trabajo, se utiliza una llamada Fotometr\'ia de Apertura. \'Esta consiste en tomar la imagen de la estrella y elegir un radio de apertura, el cual define un c\'irculo dentro del cual debiese estar todo el flujo de la estrella, para integrar el brillo proveniente de ese c\'irculo (es decir, sumar la informaci\'on de los electrones excitados en ese pixel del CCD). Con esto, se obtiene el flujo de la luz proveniente de toda la estrella. Adem\'as del radio de apertura, se elige un anillo alrededor de la estrella (caracterizado por dos radios) el cual se usa como correcci\'on, ya que el cielo alrededor de la estrella no es completamente negro en la imagen captada.


El objetivo de este trabajo es armar desde cero las herramientas computacionales que se utilizan para hacer fotometr\'ia de apertura. Se debe armar el programa que, teniendo los datos, puedan leerlos y realizar una lectura de la cantidad de luz (cantidad de electrones excitados en la placa CCD). El programa se hace en lenguaje Python 2.7, usando el paquete astropy.io para el manejo de los archivos que se entregan (de extensi\'on .fits), matplotlib para todos los gr\'aficos e im\'agenes resultantes y scipy para las funciones matem\'aticas necesarias.

\section{Desarrollo}
\subsection{Procedimiento}

Para cumplir el objetivo de este trabajo, se divide en cuatro pasos:

\begin{itemize}
	\item Creaci\'on de los master Bias y Flats: dentro de los archivos se encuentran archivos de tipo Bias (que muestran una foto en negro, de la cual se saca c\'omo es la imagen para ausencia de luz) y Flats (fotos de un domo de color parejo, para encontrar los defectos propios de la c\'amara CCD ocupada). Se necesita un promedio de ambos para corregir las im\'agenes obtenidas.
	\item Detecci\'on del centro de masa de la estrella observada: para integrar el brillo de la estrella se necesita saber d\'onde se ubica su centro. Se estima a trav\'es del centro de masa, que en este caso es el centro de donde llega m\'as brillo.
	\item Creaci\'on del perfil radial de la estrella: da la informaci\'on del flujo a cierto radio desde el centro de la estrella. De esta forma se puede determinar el radio de la estrella y lo que es cielo.
	\item Fotometr\'ia de apertura: teniendo lo anterior, se puede integrar para obtener el flujo de la estrella. 
\end{itemize}

Cada uno de estos pasos representa una funci\'on (o dos) que deben implementarse en el programa. 

\subsection{Resultados}


\section{Resultados}
La figura 1 muestra la interpolaci\'on con polinomios de Lagrange. De la figura se aprecia que \'esta aproximaci\'on es m\'as fiel a la funci\'on original a medida se aumenta la cantidad de puntos utilizados para la interpolaci\'on, sin embargo a los extremos se observa un comportamiento oscilatorio de la aproximaci\'on, conocido como el Fen\'omeno de Runge\footnote{https://en.wikipedia.org/wiki/Runge\%27s\_phenomenon}.

La figura 2 muestra la interpolaci\'on con Spline. A mayor cantidad de puntos la interpolaci\'on con Spline es mejor aproximaci\'on que la interpolaci\'on con Lagrange. La interpolaci\'on con Spline no presenta el Fen\'omeno de Runge cuando se usa una cantidad grande de puntos.

La figura 3 muestra la diferencia entre ambos m\'etodos de interpolacion y la funci\'on original. Se puede apreciar de ella que la aproximaci\'on Spline es mejor que la aproximaci\'on con Lagrange.

La figura 4 muestra la imagen de la galaxia reconstruida.
\begin{figure}[!hbpt]
\centering
\begin{subfigure}{.5\textwidth}
  \centering
  \includegraphics[width=9cm, height=7cm]{img/lagrange5.pdf}
  \caption{Interpolaci\'on con Lagrange para un intervalo\\ $[-1,1]$ equiespaciado con 5 puntos.}
\end{subfigure}%
\begin{subfigure}{.5\textwidth}
  \centering
  \includegraphics[width=9cm, height=7cm]{img/lagrange10.pdf}
  \caption{Interpolaci\'on con Lagrange para un intervalo\\ $[-1,1]$ equiespaciado con 10 puntos.}
\end{subfigure}
\begin{subfigure}{.5\textwidth}
  \centering
  \includegraphics[width=9cm, height=7cm]{img/lagrange15.pdf}
  \caption{Interpolaci\'on con Lagrange para un intervalo\\ $[-1,1]$ equiespaciado con 15 puntos.}
\end{subfigure}%
\begin{subfigure}{.5\textwidth}
  \centering
  \includegraphics[width=9cm, height=7cm]{img/lagrange20.pdf}
  \caption{Interpolaci\'on con Lagrange para un intervalo\\ $[-1,1]$ equiespaciado con 20 puntos.}
\end{subfigure}
\caption{Distintas interpolaciones utilizando polinomios de Lagrange.}
\end{figure}

\begin{figure}[!hbpt]
\centering
\begin{subfigure}{.5\textwidth}
  \centering
  \includegraphics[width=9cm, height=7cm]{img/spline5.pdf}
  \caption{Interpolaci\'on con Spline para un intervalo\\ $[-1,1]$ equiespaciado con 5 puntos.}
\end{subfigure}%
\begin{subfigure}{.5\textwidth}
  \centering
  \includegraphics[width=9cm, height=7cm]{img/spline10.pdf}
  \caption{Interpolaci\'on con Spline para un intervalo\\ $[-1,1]$ equiespaciado con 10 puntos.}
\end{subfigure}
\begin{subfigure}{.5\textwidth}
  \centering
  \includegraphics[width=9cm, height=7cm]{img/spline15.pdf}
  \caption{Interpolaci\'on con Spline para un intervalo\\ $[-1,1]$ equiespaciado con 15 puntos.}
\end{subfigure}%
\begin{subfigure}{.5\textwidth}
  \centering
  \includegraphics[width=9cm, height=7cm]{img/spline20.pdf}
  \caption{Interpolaci\'on con Spline para un intervalo\\ $[-1,1]$ equiespaciado con 20 puntos.}
\end{subfigure}
\caption{Distintas interpolaciones utilizando Spline.}
\end{figure}

\fig{lagrange_spl_diff.pdf}{18cm}{13cm}{Diferencia entre la funci\'on original y las aproximaciones de Lagrnge y Spline para 50 puntos en el intervalo $[-1, 1]$.}
\fig{fixed_galaxy.pdf}{18cm}{13cm}{Imagen arreglada de la galaxia.}

\newpage

\section{Conclusiones}
Se concluye que una interpolaci\'on con Spline c\'ubico es una mejor aproximaci\'on a mayor cantidad de puntos que una interpolaci\'on con un polinomio de Lagrange, o si es de inter\'es analizar los extremos de la funci\'on a interpolar ya que, a diferencia de \'esta, Spline no presenta el Fen\'omeno de Runge.

Se concluye que la interpolaci\'on es una herramienta \'util en el an\'alisis de im\'agenes, en particular si se busca reparar una imagen da\~nada.

\end{document}
